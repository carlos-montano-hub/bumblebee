Para asegurar la salud y producción eficiente de las colmenas es necesario contar con un monitoreo constante, este proceso se realiza de manera manual, con un apicultor en sitio, que visita la colmena en un periodo de 8 a 15 días en búsqueda de anomalías y enfermedades \cite{correa_benitez_2018}.

Conforme se incrementa la producción de miel, el proceso de monitoreo manual se vuelve ineficiente y tardado, añadiendo requerimientos de logística y el riesgo de que no se monitoreen las colmenas en el tiempo recomendado.

Abordar este problema brinda la posibilidad de obtener beneficios de logística en la producción apícola, así como de brindar una perspectiva general de todas las colmenas. Sin embargo, lo anterior plantea las siguientes interrogantes:

\begin{enumerate}
    \item ¿Cómo implementar un sistema de monitoreo portátil de temperatura, humedad, peso y audio que sea efectivo y compatible con las necesidades específicas de las colmenas apícolas en Sonora?
    \item ¿Cómo implementar un sistema de reconocimiento de patrones que procese los datos recopilados por el sistema de monitoreo y reconozca eventos de interés?
    \item ¿Qué características y funcionalidades son necesarias para el desarrollo de un servicio que procese los datos recopilados por el sistema de monitoreo?
    \item ¿Cómo se puede diseñar una aplicación móvil que brinde una interfaz para visualizar los datos recopilados por el sistema de monitoreo y los resultados del sistema de reconocimiento de patrones?
\end{enumerate}

De lo anterior, se define la siguiente problemática:

¿Qué funcionales debe considerar la implementación de un sistema móvil de monitoreo, procesamiento y visualización de las variables de temperatura, humedad, peso y audio de colmenas apícolas, así como el uso de un algoritmo de inteligencia artificial, que permita eficientar el proceso de polinización?