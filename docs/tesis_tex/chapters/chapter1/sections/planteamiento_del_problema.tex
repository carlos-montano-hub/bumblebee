Para asegurar la salud y producción eficiente de las colmenas es necesario contar con un monitoreo constante, este proceso se realiza de manera manual, con un apicultor en sitio, que visita la colmena en un periodo de 8 a 15 días en búsqueda de anomalías y enfermedades \cite{correa_benitez_2018}.

Conforme se incrementa la producción de miel, el proceso de monitoreo manual se puede volver ineficiente y tardado, añadiendo requerimientos de logística y el riesgo de que no se monitoreen las colmenas en el tiempo recomendado.

Abordar este problema podría significar en beneficios de logística en la producción apícola, así como brindar una perspectiva general de todas las colmenas.