\chapter*{Introducción}
\addcontentsline{toc}{chapter}{Introducción}

La apicultura es la actividad que se encarga de la crianza y explotación de abejas, de la cual dependen miles apicultores a nivel mundial, y más de 14 mil apicultores a nivel nacional \cite{data_mexico_2023a}.
Las colmenas apícolas tienen 2 funciones, la primera es la elaboración de productos provenientes de las abejas, entre estos se encuentra el polen, los propóleos, la cera, la jalea real y, el principal producto, la miel. Además, las colmenas brindan un servicio de polinización a las plantas con flor que se encuentren a los alrededores de la colmena \cite{bradbear_2005}.
Como ya se mencionó una de las funciones de las colmenas es la polinización, este proceso es fundamental para la reproducción de plantas con flor, y se realiza al momento en el que una abeja lleva polen de la parte masculina de una planta a una parte femenina \cite{bradbear_2005}.
