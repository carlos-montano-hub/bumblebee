\begin{center}
\large{Resumen}
\end{center}

    La apicultura es una actividad esencial para la producción de miel y otros productos derivados de las abejas, así como para la polinización de plantas. En México, la apicultura juega un papel significativo, siendo el noveno productor de miel a nivel mundial. El estado de Sonora, aunque contribuye con solo el 7\% de la producción nacional de miel, destaca por la utilización de sus colmenas, principalmente para la polinización de cultivos. Para asegurar la salud y producción eficiente de las colmenas, es necesario contar con un monitoreo constante, que tradicionalmente se realiza de manera manual, con un apicultor visitando la colmena cada 8 a 15 días en búsqueda de anomalías y enfermedades. Con el aumento de colmenas, este proceso manual puede volverse ineficiente y tardado, añadiendo requerimientos de logística y el riesgo de no monitorear las colmenas en el tiempo recomendado. Esta investigación propone el desarrollo de un sistema de monitoreo de colmenas apícolas utilizando técnicas de Internet de las Cosas (IoT), con el fin de mejorar la eficiencia y efectividad del monitoreo. El sistema integrará sensores para medir variables como temperatura, humedad, peso y audio, y empleará un algoritmo de reconocimiento de patrones para identificar eventos de interés que puedan requerir la atención del apicultor. Se desarrollará una aplicación móvil para la visualización de los datos recopilados y los resultados del procesamiento. Este sistema tiene como objetivo proporcionar a los apicultores información en tiempo real sobre el estado de sus colmenas, mejorando así la toma de decisiones y optimizando el manejo de las colmenas.